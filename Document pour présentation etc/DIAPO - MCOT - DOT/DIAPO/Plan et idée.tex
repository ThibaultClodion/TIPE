\documentclass[12pt]{article}

\title{Plan et idée pour Diapo}
\author{Thibault Clodion}

\begin{document}

\maketitle % Permet d'afficher le titre, l'author etc

\section{Introduction}

\begin{itemize}
    \item Enjeu (exemple: 11 Septembre)
    \item Lien avec le thème de l'année : La ville
\end{itemize}

\section{Modélisation de la Foule}

\begin{itemize}
    \item L'idée est de simplifier le réel mais de rester mais de rester cohérent car les simulations 
    doivent coller à des observations réels : Rester simple mais cohérent
    \item Pourquoi on va supposer que chaque personne connait le pcc(PSO)
    \item Choix de A* :
    \newline - Simplicité, coût faible(200 personnes), déjà implémenté en Unity.
    \newline - Double calcul : pcc puis intelligement (fonction Move)
    \item Simplification autres : (cône de même taille, même vitesse, ...) (ce qui en soit aurait une influence si on les changeaient).
\end{itemize}

\section{Normes et choix du type de Bâtiment}

\begin{itemize}
    \item Normes XF-35 : une première base
    \item Taille du bâtiment, type du bâtiment
    \item Nombre de personnes et contraintes imposées(Bureaux PDG, employés proches...)
    \item Choix taille du mobilier et du mobilier présent
\end{itemize}

\section{Optimisation du bâtiment}

\begin{itemize}
    \item 10 Premiers Bâtiments pour observations
    \item Hypothèses découlant des observations
    \item Création d'un nouveau bâtiment "assez aléatoire" et optimisation selon Hypothèses
    \item Micro-Optimisation selon le bâtiment et donc non appliquable dans les cas généraux. (et leur influence)
    \item Conclusion sur l'efficacité de l'optimisation et les hypothèses eventuellement réappliquable pour chaque bâtiment.
\end{itemize}

\section{Conclusion}

\begin{itemize}
    \item Un temps précieux
    \newline(Montré ce que le temps gagné représente en termes de vies sauvés)
    \newline(On pourrait utiliser le traitement de données de Mme.Martin pour voir ce que n secondes en plus represente sur le nombre de morts)
    \newline(Eventuellement donné le pourcentage de vie sauvés que cela représente)
    \item Une ville plus sécurisé à l'heure où les bâtiments sont de plus en plus grand et de plus en plus nombreux.
\end{itemize}

\end{document}