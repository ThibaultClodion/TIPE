\documentclass[12pt]{article}

\usepackage{graphicx}
\graphicspath{{../Fichier_Image}}
\title{Hypothèse 3.(c) Porte Bureaux Regulation}
\author{Thibault Clodion}

\begin{document}

\maketitle % Permet d'afficher le titre, l'author etc

\underline{Hypothèse :} Certain bureaux (même grand) doivent avoir qu'une seule porte pour réguler les flux
\newline\newline
\underline{Expérience :}Des simulations avec le bâtiment en photo (le 7.), une où il a une porte (comme sur la photo) et une où il y en a deux dont une à gauche, donc il y aura plus de régulation des flux
\newline\includegraphics[scale=0.17]{7. bureau 1 porte.png}\newline
\newline
(peut-être le temps qu'ils fassent le tour revient à diviser le flux à voir.)
\newline\newline

Remarque : j'ai changé un peut le bâtiment 7 en modifiant les portes en bleu. Cela permet d'éviter des colisions qui biaiseraient les résultats (car le fait qu'il y est qu'une porte
aurait créer des collisions, c'est d'ailleurs un changement intéressant pour le 6.)
\newline
\includegraphics[scale=0.3]{3.(c) 7. avec 1 porte.png}
\newline\newline

3.(c) 7. avec 1 porte
\newline\newline
Temps moyen de dernière sortie :
\newline
$\hspace*{0.2cm}$- Les personnes dans le bureaux ciblé rentrent dans le couloir par la droite et se retrouve donc peu en collision avec le reste du flux a gauche
(le phénomène se voit sur l'image au dessus)
\newline
$\hspace*{0.2cm}$- On a une régulation du flux qui permet d'éviter le nombre de collisions
\newline\newline

3.(c) 7. avec 2 portes
\newline\newline
\includegraphics[scale=0.3]{3.(c) 7. avec 2 portes.png}
\newline\newline
Temps moyen de dernière sortie :
\newline
$\hspace*{0.2cm}$- La porte ajouté, créer en effet plus de flux du côté gauche du bureau
\newline
$\hspace*{0.2cm}$- Il faut donc voir si l'importance du flux générer augmente en effet le temps
\newline\newline

\section{Résultat}


\end{document}