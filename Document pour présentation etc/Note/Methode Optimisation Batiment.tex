\documentclass[12pt]{article}

\title{Méthode Optimisation Batiment}
\author{Thibault Clodion}

\begin{document}

\maketitle % Permet d'afficher le titre, l'author etc

Chaque Phase (chapitre) doit avoir son propre document latex pour noter les observations et ce que j'ai effectué durant cette phase.

\section{10 Premiers Batiments}

L'idée est de faire 10 premiers batiments en guise de "test" pour voir :
\begin{itemize}
    \item Si la simulation fonctionne bien et est cohérente
    \item Il y a t'il des premières observations qui permettent de tirer des hypothèses
    sur les paramètres qui font influer le temps de sortie ?
\end{itemize}

\section{Hypothèse pour optimiser le bâtiment}

L'idée est ici de continuer à faire des expériences ayant pour but de valider ou refuter des hypothèses.
(normalement les 10 premiers batiments donnent quelque hypothèses et il est pas impossible que d'autre viennent après)
\newline\newline
C'est en réalité ici que ce joue la grosse partie de l'optimisation car ça va me permettre de mieux comprendre comment optimiser un batiment.
Madame Le Gluher m'a aussi donné l'idée de faire plus tard un nouveau batiment (nouveau cahier des charges), pour voir si ces hypothèses
fonctionnent de manière globale, dans ce cas mon résultat sera encore plus général.
\newline\newline
C'est à dire que l'idée est de faire :
\begin{itemize}
    \item Emettre une hypothèse sur l'influence de quelque chose sur le temps d'évacuation
    \item Mettre en place les expériences permettant de répondre à mon hypothèse
    \item S'il est réfuté alors je pourrais la mettre de côté, sinon la garder pour pouvoir optimiser dans le futur
    le bâtiment.
\end{itemize}

A l'issue de cette phase je devrais avoir normalement une idée de ce qui influe sur le temps d'évacuation d'un batiment
et ainsi je pourrais construire un batiment qui respecte ces hypothèses pour qu'il soit plus optimisé. (il est pas impossible que je trouve
dans la phase trois de nouvelles hypothèses, je pourrais éventuellement les rajouter ici se sera plus simple pour la présentation).

\section{Création du Bâtiment}

\begin{enumerate}
    \item Construire un nouveau bâtiment qui soit "assez" aléatoire, du moins pas optimisé.
    \item Lui appliquer les hypothèses qui permettent de diminuer le temps de sortie et voir si en effet elles font diminuer le temps de sortie moyen.
    \item Faire un premier constat de l'amélioration et de la validité ou non des hypothèses faite dans le 2.
    \item Observer le bâtiment obtenu et voir si il est encore possible de lui proposer des micro-optimisation qui sont trop minimes et spécifiques pour être des hypothèses à part entière.
    \item Le bâtiment étant fini, on peut alors conclure sur son amélioration et sur le temps précieux gagné (on pourra également le comparé aux bâtiments des hypothèses éventuellement)
    
    (Il faut espérer que les hypothèses faite ferront en effet diminuer le temps de sortie, sinon c'est la merde car il faudra soit faire une conclusion sur le fait que c'est un semi-echec
    soit faire un nouveau bâtiment.)
\end{enumerate}


\end{document}