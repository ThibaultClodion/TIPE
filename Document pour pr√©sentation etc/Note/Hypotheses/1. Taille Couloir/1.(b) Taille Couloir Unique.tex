\documentclass[12pt]{article}

\usepackage{graphicx}
\graphicspath{{../Fichier_Image}}

\title{Hypothèse 1.(b) Taille Couloir Unique}
\author{Thibault Clodion}

\begin{document}

\maketitle % Permet d'afficher le titre, l'author etc

\underline{Hypothèse :} La taille de tous les couloirs doit être unique (pas d'effet entonnoir)
\newline\newline
\underline{Expérience :} Je pars pour cette expérience du bâtiment 10. (des 10 premiers batiments) car c'est celui ayant le meilleur temps et où les couloirs autres que le principal sont le plus emprunté.
\newline\newline
Je garde le couloir principal selon le 1.a (donc 1m) et je change la taille des autres couloirs en 80cm, 1m, 1m25, 1m50, 2m.
\newline
Une autre expérience où je change tous les couloirs : 80cm, 1m, 1m25, 1m50, 2m. (pour vérifier si l'unicité est importante)
\newline\newline

1.(b) Couloir Autre 80cm :
\newline\newline

1.(b) Couloir Autre 1m : 23.20 s
\newline\newline

1.(b) Couloir Autre 1m25 :
\newline\newline

1.(b) Couloir Autre 1m50 :
\newline\newline

1.(b) Couloir Autre 2m :
\newline\newline

1.(b) Couloirs 80cm :
\newline\newline

1.(b) Couloirs 1m : 23.20 s
\newline\newline

1.(b) Couloirs 1m25 :
\newline\newline

1.(b) Couloirs 1m50 :
\newline\newline

1.(b) Couloirs 2m :
\newline\newline

\end{document}