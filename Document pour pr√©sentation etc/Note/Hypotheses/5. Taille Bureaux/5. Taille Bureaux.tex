\documentclass[12pt]{article}

\usepackage{graphicx}
\graphicspath{{../Fichier_Image}}

\title{Hypothèse 5.Taille Bureaux}
\author{Thibault Clodion}

\begin{document}

\maketitle % Permet d'afficher le titre, l'author etc

\underline{Hypothèse :} La taille des bureaux influe sur le temps de sortie (comme vu exp 1. +
4.-7.)
\newline\newline
\underline{Observations déjà faite :}
\newline
$\hspace*{0.2cm}$- Il ne faut pas contraindre la taille des bureaux a des bureaux trop petits (exp 4.) cela fait qu'il y a peu de chemins
et que tout le monde se retrouve à se pousser dans les couloirs
\newline
$\hspace*{0.2cm}$- Il ne faut pas avoir de trop gros bureaux (1.) cela génère de trop gros flux à leur sortie.(on se retrouve presque dans les premières simulations 250 desk uniforme)
\newline
$\hspace*{0.2cm}$- En ayant des bureaux de tailles raisonnables (7.) on obtient un meilleur temps car à la fois on divise les flux importants en sortie de bureau
et on peut en même temps orienté les flux plus facilement.
\newline\newline
\underline{Expérience :}
\newline
5, 10, 15, 20, 25, 30, 35, 40, 45, 50, 55 personnes par bureaux (au max car je peux pas faire tous bureaux taille unique mais on pousse au maximum la taille) 
pour essayer d'observer une gaussienne et conclure sur le nombre de personnes par bureaux qu'il faut au max.
\newline\newline

5. 5 personnes max
\newline\newline
Temps moyen de dernière sortie :
\newline
$\hspace*{0.2cm}$-
\newline
$\hspace*{0.2cm}$-
\newline\newline

5. 10 personnes max
\newline\newline
Temps moyen de dernière sortie :
\newline
$\hspace*{0.2cm}$-
\newline
$\hspace*{0.2cm}$-
\newline\newline

5. 15 personnes max
\newline\newline
Temps moyen de dernière sortie :
\newline
$\hspace*{0.2cm}$-
\newline
$\hspace*{0.2cm}$-
\newline\newline

5. 20 personnes max
\newline\newline
Temps moyen de dernière sortie :
\newline
$\hspace*{0.2cm}$-
\newline
$\hspace*{0.2cm}$-
\newline\newline

5. 25 personnes max
\newline\newline
Temps moyen de dernière sortie :
\newline
$\hspace*{0.2cm}$-
\newline
$\hspace*{0.2cm}$-
\newline\newline

5. 30 personnes max
\newline\newline
Temps moyen de dernière sortie :
\newline
$\hspace*{0.2cm}$-
\newline
$\hspace*{0.2cm}$-
\newline\newline

5. 35 personnes max (même que 30 personnes max)
\newline\newline

5. 40 personnes max
\newline\newline
Temps moyen de dernière sortie :
\newline
$\hspace*{0.2cm}$-
\newline
$\hspace*{0.2cm}$-
\newline\newline

5. 45 personnes max (même que 40 personnes max)
\newline\newline

5. 50 personnes max
\newline\newline
Temps moyen de dernière sortie :
\newline
$\hspace*{0.2cm}$-
\newline
$\hspace*{0.2cm}$-
\newline\newline

5. 55 personnes max
\newline\newline
Temps moyen de dernière sortie :
\newline
$\hspace*{0.2cm}$-
\newline
$\hspace*{0.2cm}$-
\newline\newline

\end{document}