\documentclass[12pt]{article}

\usepackage{graphicx}
\graphicspath{{../Fichier_Image}}

\title{Hypothèse 7. Disposition Meuble}
\author{Thibault Clodion}

\begin{document}

\maketitle % Permet d'afficher le titre, l'author etc

\underline{Hypothèse :} La disposition des meubles influent sur le temps de sortie.
\newline\newline
\underline{Expérience :}
\newline


\section{Resultats}

C'est beaucoup trop dur d'évaluer l'influence de la disposition des meubles car :
\newline - Parfois bloquer les issues ralentit les personnes, ce qui fait que les flux sont moins importants donc les sorties plus rapides
\newline - Parfois le faite de modifier la disposition des meubles font que les personnes ne passent plus par le chemin ce qui finit par réduire ou pas le temps de sorties.
\newline - Donc ce qui semble avoir des effets négatifs à parfois des effets positifs au final, donc c'est beaucoup trop chaotique pour être précisement étudier.
\newline\newline

Je propose donc de régler l'emplacement des meubles seulement sur le bâtiment optimisé que l'on cherchera à créer.
\newline\newline

Il conviendra de dire à l'oral que finalement cette hypothèse est difficile à vérifier et que en vu des articles (sur le faite que l'on ne peut pas conclure sur l'effet d'un obstacle)
alors il vaut mieux laisser tomber cette hypothèse. Cela peut néanmoins être intéressant de mentionner le fait que j'ai essayé de le présenter.


\end{document}