\documentclass[12pt]{article}

\usepackage{graphicx}
\graphicspath{ {Fichier_Image/} }

\title{Hypothèses}
\author{Thibault Clodion}

\begin{document}

\maketitle % Permet d'afficher le titre, l'author etc

L'idée est ici de continuer à faire des expériences ayant pour but de valider ou refuter des hypothèses.
Après cela je pourrais mieux comprendre comment optimiser un bâtiment et adapter les résultats à un autre Cahier de Charge éventuellement.

\section{Hypothèses au propre}

\begin{enumerate}
    \item La taille des couloirs influe sur le temps de sortie
    \begin{enumerate}
        \item La taille des couloirs menant aux portes de sorties (juste devant) doit être la même que celle de la porte de sortie
        \item La taille de tous les couloirs doit être unique (pas d'effet entonnoir)
    \end{enumerate}

    \item 
    
    \item Le nombre de porte par bureau influe sur le temps de sorties
    \begin{enumerate}
        \item Les bureaux du milieux ont besoin de plusieurs portes pour diminuer le temps de sorties
        \item Les petits bureaux doivent avoir qu'une seule porte pour diminuer le temps de sortie
        \item Certain bureaux (même grand) doivent avoir qu'une seule porte pour réguler les flux
    \end{enumerate}

    \item
    
    \item La taille des bureaux influe sur le temps de sortie (comme vu exp 1. + 4.-7.)
    
    \item Les croisements (carrefour) augmente le temps de sortie si il génère des flux opposés
    
    \item La disposition des meubles influent sur le temps de sortie.

\end{enumerate}

\section{Expériences + Observations + Conclusions}

Voir les autres fichiers

\section{Hypothèses gardé pour optimiser le Bâtiment final}

\begin{enumerate}
    \item (1.b) La taille des couloirs doit être unique et un peu plus grande que celle des portes de sorties principales. (1m25 mieux trouvé)
    \item (3.a) Les bureaux proches du centre doivent avoir plusieurs portes (mais raisonnable).
    \item (3.b) Les petits bureaux ne doivent avoir qu'une porte (c'est très négligeable à voir si je garde).
    \item (5.) Les bureaux doivent être de taille comprise entre 20 et 50 personnes.
    \item (6.) Il est nécessaire d'éviter les croissements (pour couloirs et portes)
\newline\newline\end{enumerate}

Remarque :
\begin{enumerate}
    \item Donne la taille des couloirs
    \item 
    \item Donne le nombre de portes par bureaux
    \item 
    \item Donne la taille des bureaux
    \item Donne la disposition des bureaux, de leur porte et des couloirs.
    \item Donne l'organisation interne des bureaux. (or l'hypothèse est pas concluante)
\end{enumerate}

Donc au final cela donne une bonne idée de comment arranger le bâtiment pour qu'il soit optimisé.

\end{document}