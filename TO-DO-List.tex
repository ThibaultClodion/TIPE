\documentclass{article}

% Package pour faire les do-to list
\usepackage{enumitem,amssymb}
\newlist{todolist}{itemize}{2}
\setlist[todolist]{label=$\square$}
\usepackage{pifont}
\newcommand{\cmark}{\ding{51}}%
\newcommand{\xmark}{\ding{55}}%
\newcommand{\done}{\rlap{$\square$}{\raisebox{2pt}{\large\hspace{1pt}\cmark}}%
\hspace{-2.5pt}}
\newcommand{\wontfix}{\rlap{$\square$}{\large\hspace{1pt}\xmark}}

\title{TO DO LIST - TIPE}
\author{Thibault Clodion}

\begin{document}

\maketitle % Permet d'afficher le titre, l'author etc

\begin{todolist}
    \item[\done] Mettre en place un moyen de faire des simulations où l'on peut placer les gens (pour modélisation batiment D ce sera plus réaliste)
    \item Trouver moyen de faire descendre et monter des escaliers (pour l'instant que des bugs, même en modifiant le collider de la capsule, le rayon
    et en fesant en sorte que les escaliers soient des endroits de 'jump')
    \item Optimiser un bâtiment réel (Batiment D)
    \item Faire la MCOT.
    \item Faire la DOT.
    \item Faire le Diapo. (Regarder Google Slide)
\end{todolist}

\end{document}