\documentclass{article}

% Package pour faire les do-to list
\usepackage{enumitem,amssymb}
\newlist{todolist}{itemize}{2}
\setlist[todolist]{label=$\square$}
\usepackage{pifont}
\newcommand{\cmark}{\ding{51}}%
\newcommand{\xmark}{\ding{55}}%
\newcommand{\done}{\rlap{$\square$}{\raisebox{2pt}{\large\hspace{1pt}\cmark}}%
\hspace{-2.5pt}}
\newcommand{\wontfix}{\rlap{$\square$}{\large\hspace{1pt}\xmark}}

\title{TO DO LIST - TIPE}
\author{Thibault Clodion}

\begin{document}

\maketitle % Permet d'afficher le titre, l'author etc

\begin{todolist}
    \item[\done] Finaliser les hypothèses.
    \item Optimiser le bâtiment hypothétique.
    \item Pour valider les hypothèses : Pourquoi pas les appliquer sur un vrai bâtiment. (BFM)
    \item On pourrait aussi se proposer d'optimiser un bâtiment réel. (BFM)
    \item Tout le code est à mettre en français. (pour présentation)
    \item Faire la DOT.
    \item Faire la MCOT.
    \item Faire le Diapo.
\end{todolist}

\end{document}